\documentclass[a4paper,12pt]{article}
\usepackage[utf8]{inputenc}
\usepackage[english]{babel}
\usepackage{authblk}
\usepackage{graphicx}
\usepackage{mathptmx}
\usepackage[singlespacing]{setspace}
\usepackage[headheight=1in,margin=1in]{geometry}
\usepackage{fancyhdr}
\usepackage{amsmath}
\usepackage{booktabs}
\usepackage{hyperref}
\usepackage{caption}
\captionsetup{font=small}

\renewcommand{\headrulewidth}{0pt}
\pagestyle{fancy}

\makeatletter
\def\@maketitle{%
  \newpage
  \begin{center}%
  \let \footnote \thanks
    {\LARGE \@title \par}%
  \end{center}%
  \par
  \vskip 0.1em}
\makeatother

\chead{}

\title{The Geometry of Gridlock: Tracking Congressional Polarization\\ with Graph Attention Networks}

\date{}

\begin{document}

\maketitle
\thispagestyle{fancy}

\begin{center}
{Keywords: congressional polarization, graph attention networks, spectral analysis, co-voting networks, computational social science}
\newline
\end{center}

\setlength{\parindent}{0pt}
\setlength{\parskip}{6pt}

\section*{Extended Abstract}

Congressional polarization is typically measured by placing legislators on an ideological spectrum via DW-NOMINATE scaling (Poole \& Rosenthal, 1985; Poole \& Rosenthal, 2017). This approach captures positional divergence but misses the relational architecture of legislative cooperation: who works with whom, and how those cooperative structures fracture over time. Network approaches have demonstrated that partisan clustering increases over time (Waugh et al., 2009) and that cross-party edges have nearly vanished (Andris et al., 2015), but these analyses remain descriptive. We move beyond description by constructing co-voting networks for every U.S.\ House Congress from the 100th (1987) through the 118th (2025) and training a predictive model that learns from both network structure and temporal evolution. Using roll-call records from Voteview (Lewis et al., 2023), we compute pairwise agreement scores for all House members within each Congress and threshold at $\tau = 0.5$ to create co-voting graphs comprising approximately 442 nodes per Congress and 8,400 member-Congress observations total. We extract the algebraic connectivity (Fiedler value $\lambda_2$) from each graph's normalized Laplacian as a measure of structural integration (Fiedler, 1973) and train a Graph Attention Network (Veli\v{c}kovi\'{c} et al., 2018) augmented with temporal attention across the congressional sequence. The model takes eight-dimensional node features including DW-NOMINATE scores, party affiliation, and network-derived agreement statistics, and jointly addresses polarization trajectory forecasting, coalition detection from learned network embeddings, and defection prediction for individual legislators. Training uses Congresses 104 through 114 under a strict temporal split, with Congresses 115 through 117 held out for evaluation. We additionally introduce two novel structural indices: the Structural Resilience Index (SRI), which quantifies a network's capacity to recover from exogenous shocks by comparing pre- and post-shock Fiedler values, and the Bridge Legislator Index (BLI), which measures each member's contribution to network connectivity through the Fiedler value change upon their removal. The Fiedler value rises from 0.534 in the 100th Congress to 0.843 in the 107th (post-9/11) before collapsing to 0.032 in the 118th, representing a 94\% decline in algebraic connectivity (Figure 1, Table 1). This trajectory is sharply non-monotonic: DW-NOMINATE party distance increases smoothly over the same period, missing both the post-9/11 cooperation surge and the abruptness of the structural collapse. A counterintuitive pattern emerges from the density analysis. Network density actually increases from 0.453 to 0.493 even as the Fiedler value collapses, revealing that polarization is not a loss of cooperation but a structural rewiring in which within-party connections densify while cross-party bridges vanish. The 103rd Congress contained 429 cross-party edges; the 114th contained zero (Figure 2). An interrupted time series analysis identifies the Tea Party wave of 2010 as the single largest structural shock in the dataset: a Fiedler drop of 0.435 in one electoral cycle, accounting for 87\% of the total decline. The SRI formalizes a critical asymmetry between historical shocks. After 9/11, bipartisan national security legislation temporarily rewired the co-voting graph and pushed the Fiedler value to 0.843, but within two Congresses connectivity returned to its prior level (SRI = 1.00). After the Tea Party wave drove the Fiedler value to 0.073, no subsequent event produced comparable recovery: not the Trump presidency, not January 6th, not a global pandemic. The Fiedler value continued declining to 0.010 (SRI = 0.02), indicating that the network has transitioned from a regime capable of absorbing perturbations to one in which shocks produce permanent structural damage. On held-out Congresses, the GAT achieves a mean AUC of 0.908 for defection prediction, outperforming a Random Forest baseline of 0.852 that uses the same DW-NOMINATE features (Figure 3, Table 2). The baseline wins on the easier 115th Congress (0.968 versus 0.945), but the GAT generalizes better to later sessions as polarization dynamics shift: 0.945 versus 0.805 on the 116th and 0.835 versus 0.784 on the 117th. Applied to the 2023 McCarthy Speaker crisis, the model identifies 12 of 20 Republican holdouts from network position alone, only 3 of whom would have been flagged by a standard ideological distance rule. The case of Matt Gaetz illustrates the model's value: his DW-NOMINATE score placed him just 0.095 from the Republican median, making him appear unremarkable by ideological measures, yet the GAT assigned him a defection probability of 0.782 from his network position nine months before he led the motion to vacate the chair. The eight missed members were either freshmen lacking prior voting history or members whose rebellion was driven by personal loyalty dynamics rather than policy-based network position. The BLI analysis reveals that network connectivity is maintained by a thin layer of cross-partisan bridge legislators who are systematically eliminated through primaries, redistricting, and strategic retirement (Figure 4). The 104th Congress had 25 members above the BLI threshold; the 117th had only 11, and its top five scorers all left Congress within two years. Their selective elimination accelerates the structural disconnection documented in the spectral analysis. Counterfactual perturbation experiments quantify this concentration: reintroducing 1993-level cross-party cooperation among just 40 members would restore the 118th Congress's Fiedler value from 0.032 to 0.21, while removing the top 20 bridge legislators from the 103rd Congress drops its Fiedler value from 0.230 to below 0.08. The structural difference between eras is concentrated in fewer than 10\% of House members, suggesting that preserving even a modest number of cross-party cooperative relationships could have outsized effects on legislative integration. This work contributes the longest spectral analysis of congressional co-voting networks to date spanning 19 Congresses and 36 years, a temporal GAT architecture that jointly models structural and dynamic features of legislative networks, the SRI and BLI as interpretable measures of institutional resilience and individual structural importance, and a falsifiable forward prediction for the 119th Congress (Fiedler $\approx$ 0.028). These findings reframe polarization from an ideological phenomenon to a topological one and demonstrate that graph-based methods reveal dynamics that spatial models systematically miss.

\section*{References}

\begin{small}
\begin{itemize}
\itemsep1pt
\item[\tiny$\bullet$] Andris, C., Lee, D., Hamilton, M. J., Martino, M., Gunning, C. E., \& Selden, J. A. (2015). The rise of partisanship and super-cooperators in the U.S. House of Representatives. \textit{PLoS ONE}, \textit{10}(4), e0123507.
\item[\tiny$\bullet$] Fiedler, M. (1973). Algebraic connectivity of graphs. \textit{Czechoslovak Mathematical Journal}, \textit{23}(2), 298-305.
\item[\tiny$\bullet$] Lewis, J. B., Poole, K., Rosenthal, H., Boche, A., Rudkin, A., \& Sonnet, L. (2023). \textit{Voteview: Congressional roll-call votes database}.
\item[\tiny$\bullet$] Poole, K. T., \& Rosenthal, H. (1985). A spatial model for legislative roll call analysis. \textit{American Journal of Political Science}, \textit{29}(2), 357-384.
\item[\tiny$\bullet$] Poole, K. T., \& Rosenthal, H. (2017). \textit{Ideology and Congress}. Routledge.
\item[\tiny$\bullet$] Veli\v{c}kovi\'{c}, P., Cucurull, G., Casanova, A., Romero, A., Li\`{o}, P., \& Bengio, Y. (2018). Graph attention networks. \textit{International Conference on Learning Representations}.
\item[\tiny$\bullet$] Waugh, A. S., Pei, L., Fowler, J. H., \& Mucha, P. J. (2009). Party polarization in Congress: A social networks approach. \textit{arXiv preprint}, arXiv:0907.3509.
\end{itemize}
\end{small}

% Cited order: Figure 1, Table 1, Figure 3, Table 2, Figure 4

\begin{figure}[htp]
\centering
\includegraphics[width=\textwidth]{fig2_polarization.pdf}
\caption{Polarization over time measured by network structure (Fiedler value) and ideological distance (DW-NOMINATE party distance). The Fiedler value reveals a post-9/11 cooperation surge and a dramatic structural collapse after the Tea Party wave that the smoother ideological distance measure misses.}
\label{fig:polarization}
\end{figure}

\begin{table}[htp]
\centering
\caption{Congressional polarization metrics across selected Congresses. The Fiedler value captures network connectivity; party distance measures ideological divergence.}
\begin{tabular}{lcccccc}
\toprule
Congress & Years & Members & Fiedler & Party Dist. & Density & Mean Degree \\
\midrule
100th & 1987-89 & 439 & 0.534 & 0.643 & 0.453 & 198.5 \\
103rd & 1993-95 & 441 & 0.230 & 0.701 & 0.498 & 219.0 \\
107th & 2001-03 & 436 & 0.843 & 0.786 & 0.383 & 166.5 \\
112th & 2011-13 & 442 & 0.073 & 0.862 & 0.490 & 216.4 \\
114th & 2015-17 & 436 & 0.010 & 0.874 & 0.493 & 214.7 \\
117th & 2021-23 & 447 & 0.086 & 0.881 & 0.465 & 208.0 \\
118th & 2023-25 & 451 & 0.032 & 0.893 & 0.479 & 216.0 \\
\bottomrule
\end{tabular}
\label{tab:polarization}
\end{table}

\begin{figure}[htp]
\centering
\includegraphics[width=\textwidth]{fig1_network.pdf}
\caption{Co-voting networks for the 114th Congress (left) and 103rd Congress (right). Red edges indicate cross-party agreement above the 50\% threshold. The near-absence of red edges in the 114th visualizes the structural disconnection quantified by the Fiedler value collapse.}
\label{fig:network}
\end{figure}

\begin{figure}[htp]
\centering
\includegraphics[width=0.75\textwidth]{fig4_roc.pdf}
\caption{ROC curves for defection prediction on held-out Congresses (115th-117th). The GAT achieves a mean AUC of 0.908, with strong discrimination across all three test sessions.}
\label{fig:roc}
\end{figure}

\clearpage
\begingroup
\centering
\captionof{table}{Defection prediction performance (AUC) on held-out Congresses. The GAT outperforms the Random Forest baseline on the 116th and 117th, demonstrating superior generalization as polarization dynamics shift.}
\label{tab:defection}
\begin{tabular}{lcc}
\toprule
Congress & GAT AUC & Baseline (RF) AUC \\
\midrule
115th (2017-19) & 0.945 & 0.968 \\
116th (2019-21) & 0.945 & 0.805 \\
117th (2021-23) & 0.835 & 0.784 \\
\midrule
Average & 0.908 & 0.852 \\
\bottomrule
\end{tabular}

\vspace{2em}

\includegraphics[width=0.75\textwidth]{fig9_bridge_index.pdf}
\captionof{figure}{Bridge legislator index over time: count of members with BLI $> 3 \times 10^{-3}$ (bars) and maximum BLI score (line). The systematic decline reflects the disappearance of cross-partisan bridge members from the House.}
\label{fig:bridge_index}
\endgroup

\end{document}
