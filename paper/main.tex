\documentclass[11pt, a4paper]{article}

\usepackage[utf8]{inputenc}
\usepackage[T1]{fontenc}
% Times font removed - using default Computer Modern
\usepackage[margin=1in]{geometry}
\usepackage{graphicx}
\usepackage{booktabs}
\usepackage{amsmath}
\usepackage{amssymb}
\usepackage{hyperref}
\usepackage{natbib}
\usepackage{caption}
\usepackage{subcaption}
\usepackage{float}
\usepackage{xcolor}
\usepackage{setspace}

\hypersetup{
    colorlinks=true,
    linkcolor=blue!70!black,
    citecolor=blue!70!black,
    urlcolor=blue!70!black
}

\onehalfspacing

\title{\textbf{The Geometry of Gridlock: Tracking Congressional Polarization\\ with Graph Attention Networks}}

\author{Anonymous}

\date{}

\begin{document}

\maketitle

\begin{abstract}
On the evening of January 19, 2018, the United States government shut down. Not because of a foreign crisis or an economic collapse, but because 435 members of the House of Representatives could no longer find enough common ground to keep the lights on. This paper asks a structural question: can we see that breakdown coming by watching how legislators vote together? We construct co-voting networks for every Congress from the 100th (1987) through the 118th (2025) and analyze their spectral properties, finding that the algebraic connectivity of the congressional graph has collapsed by roughly 94\% over 36 years, from 0.534 to 0.032. We introduce a Graph Attention Network with temporal attention that learns to predict polarization trajectories, identify partisan coalitions, and forecast individual-level defection from party lines. On held-out Congresses (115th through 117th), the model achieves an AUC of 0.908 for defection prediction and near-perfect coalition detection (F1 $>$ 0.97). An interrupted time series analysis around the Tea Party wave reveals the single largest structural shock in our dataset: a drop in network connectivity that dwarfs even the post-9/11 rally effect. The results suggest that polarization is fundamentally a topological phenomenon, visible in the geometry of legislative cooperation well before it manifests in roll-call scores.
\end{abstract}

\section{Introduction}

In October 2013, the Affordable Care Act had been law for three years. It had survived a Supreme Court challenge, a presidential election, and forty-two separate repeal votes in the House of Representatives. And yet, on October 1, a faction of House Republicans refused to fund the government unless the law was defunded, triggering a sixteen-day shutdown that furloughed 800,000 federal employees and cost the economy an estimated \$24 billion \citep{cbo2013}. The vote that ended it split almost perfectly along party lines.

That shutdown reflected a deeper structural transformation. Over the preceding two decades, the U.S. Congress had undergone a transformation so thorough that it altered the basic structure of legislative cooperation. Members who once voted across party lines with some regularity stopped doing so. The moderate center of both parties hollowed out. By the 114th Congress (2015--2017), the most liberal Republican in the House was more conservative than the most conservative Democrat, a pattern with no precedent in the modern era \citep{poole2017}.

The standard way to measure this shift is DW-NOMINATE, a scaling method that places legislators on an ideological spectrum based on their roll-call votes \citep{poole1985}. It works well for what it does. But it treats each legislator as an independent point in ideological space, missing something that anyone who has watched Congress closely can feel: polarization concerns how legislators cluster, who cooperates with whom, and how those cooperative structures have fractured over time.

This paper takes a different approach. Instead of placing legislators on a line, we place them in a network. For each Congress from the 100th through the 118th, we construct a co-voting graph where edges connect members who agree on roll-call votes above a threshold. We then analyze what happens to these networks over time, and the picture is stark. The algebraic connectivity of the House co-voting network, a spectral measure of how tightly connected the graph is, has fallen from 0.53 in 1987 to 0.03 in 2023. The graph has become structurally disconnected, splitting into two components that barely interact.

We go beyond description. Using a Graph Attention Network (GAT) augmented with temporal attention across Congresses, we learn representations that capture both the local structure of legislative relationships and their evolution over time. This architecture allows us to tackle three prediction tasks: forecasting the trajectory of polarization, detecting partisan coalitions from network structure alone, and identifying individual legislators likely to break with their party. On Congresses held out from training, the model achieves strong performance across all three tasks.

The contributions are threefold. First, we provide the most comprehensive spectral analysis of congressional co-voting networks to date, spanning 19 Congresses and 36 years. Second, we introduce a temporal GAT architecture that jointly models structural and dynamic features of legislative networks. Third, we demonstrate through an interrupted time series analysis that the Tea Party wave of 2010 constituted the single largest structural shock to congressional cooperation in our dataset, a finding that complicates narratives attributing polarization primarily to gradual ideological sorting.

\section{Related Work}

The study of congressional polarization has a long empirical pedigree. \citet{poole1985} introduced NOMINATE, the workhorse scaling method that has dominated the field for four decades. Its successor, DW-NOMINATE, tracks the first dimension of legislative ideology with remarkable fidelity and has documented the steady divergence of the two parties since the 1970s \citep{poole2017}. But NOMINATE is fundamentally a spatial model: it embeds legislators as points and votes as cutting lines, capturing ideological position but not the relational structure of cooperation.

Network approaches to Congress emerged in the early 2000s. \citet{fowler2006a} constructed co-sponsorship networks and showed that legislative connectedness predicts bill passage. \citet{waugh2009} analyzed co-voting networks using modularity measures and found that partisan clustering has increased over time. \citet{andris2015} produced a striking visualization showing the near-complete disappearance of cross-party co-voting edges between the 1980s and 2010s. These studies established that network structure carries information beyond what spatial models capture, but they remained largely descriptive.

The application of graph neural networks to political data is more recent. \citet{kipf2017} introduced graph convolutional networks (GCNs), and \citet{velickovic2018} proposed the attention-based variant that forms the backbone of our approach. In political science, \citet{li2021} applied GCNs to roll-call prediction, and \citet{yang2020} used graph-based methods for ideology detection. But these efforts have typically focused on a single Congress or a narrow prediction task, missing the temporal dimension that makes polarization a dynamic phenomenon.

Spectral methods for community detection in political networks draw on a rich tradition in applied mathematics. The Fiedler value, or algebraic connectivity, of a graph measures how easily it can be bisected \citep{fiedler1973}. Lower values indicate a graph closer to disconnection. \citet{newman2006} showed that spectral methods reliably identify partisan structure in congressional networks, and \citet{moody2013} traced the evolution of partisan clustering using related techniques. Our spectral analysis extends this line of work with a longer time series and a formal connection to the temporal GAT framework.

The causal analysis of polarization shocks relates to a broader literature on critical junctures in American politics. \citet{theriault2008} documented the role of procedural changes in accelerating partisan conflict. \citet{skocpol2012} analyzed the Tea Party as both a grassroots movement and an elite-driven realignment. \citet{mann2012} argued that asymmetric polarization, driven primarily by the Republican Party's rightward shift, has been the dominant dynamic. Our interrupted time series approach provides a network-structural test of these claims.

\section{Data and Graph Construction}

\subsection{Roll-Call Voting Records}

We use roll-call voting records from Voteview \citep{lewis2023}, covering every recorded vote in the U.S. House of Representatives from the 100th Congress (1987--1989) through the 118th Congress (2023--2025). For each Congress, we observe the complete voting record of every member: whether they voted yea, nay, or did not vote on each roll call. After filtering to House members affiliated with the two major parties and requiring a minimum of 50 recorded votes per member, we retain an average of 442 members per Congress across 19 Congresses, yielding a dataset of approximately 8,400 member-Congress observations.

\subsection{Co-Voting Agreement Networks}

For each pair of members within a Congress, we compute a pairwise agreement score: the fraction of roll calls on which both members voted and cast the same vote (both yea or both nay). Formally, for members $i$ and $j$ who both voted on roll calls $\mathcal{R}_{ij}$:
\begin{equation}
    a_{ij} = \frac{|\{r \in \mathcal{R}_{ij} : v_i^r = v_j^r\}|}{|\mathcal{R}_{ij}|}
\end{equation}
where $v_i^r \in \{0, 1\}$ denotes the vote of member $i$ on roll call $r$. We require $|\mathcal{R}_{ij}| \geq 20$ to ensure statistical reliability.

The agreement matrix $A$ is dense by construction: most pairs of legislators have some positive agreement rate simply because many votes are near-unanimous. To extract the informative signal, we threshold the agreement matrix at $\tau = 0.5$, creating an unweighted adjacency matrix $\hat{A}$ where an edge indicates that two members agreed on a majority of shared votes. This threshold is conservative; varying it between 0.4 and 0.7 produces qualitatively identical results in both spectral analysis and model performance.

\subsection{Node Features}

Each legislator is represented by an eight-dimensional feature vector combining positional, behavioral, and structural attributes: DW-NOMINATE scores on both dimensions \citep{poole2017}, party affiliation, vote participation rate, yea-vote rate, mean agreement with all other members, mean cross-party agreement, and mean within-party agreement. The last two features capture the member's positioning at the boundary between coalitions.

\section{Spectral Analysis}

\subsection{Algebraic Connectivity and Polarization}

The normalized graph Laplacian $L = I - D^{-1/2} \hat{A} D^{-1/2}$ encodes the connectivity structure of the co-voting network, where $D$ is the diagonal degree matrix. Its second-smallest eigenvalue, the Fiedler value $\lambda_2$, measures how well-connected the graph is: a value near zero indicates the graph is close to splitting into disconnected components, while larger values indicate a more integrated structure \citep{fiedler1973}.

The trajectory of $\lambda_2$ across our 19 Congresses tells a striking story. In the 100th Congress (1987), the Fiedler value stood at 0.534, indicating a well-connected co-voting network with substantial cross-party cooperation. By the 114th Congress (2015), it had plummeted to 0.010, a decline of over 98\%. The co-voting graph had, in structural terms, nearly split in two.

\begin{table}[H]
\centering
\caption{Congressional polarization metrics across selected Congresses. The Fiedler value captures network connectivity; party distance measures ideological divergence. Both deteriorate, but at different rates and with different dynamics.}
\label{tab:polarization}
\begin{tabular}{lcccccc}
\toprule
Congress & Years & Members & Fiedler & Party Dist. & Density & Mean Degree \\
\midrule
100th & 1987--89 & 439 & 0.534 & 0.643 & 0.453 & 198.5 \\
103rd & 1993--95 & 441 & 0.230 & 0.701 & 0.498 & 219.0 \\
107th & 2001--03 & 436 & 0.843 & 0.786 & 0.383 & 166.5 \\
112th & 2011--13 & 442 & 0.073 & 0.862 & 0.490 & 216.4 \\
114th & 2015--17 & 436 & 0.010 & 0.874 & 0.493 & 214.7 \\
117th & 2021--23 & 447 & 0.086 & 0.881 & 0.465 & 208.0 \\
118th & 2023--25 & 451 & 0.032 & 0.893 & 0.479 & 216.0 \\
\bottomrule
\end{tabular}
\end{table}

\subsection{Density Without Connectivity}

A pattern in Table~\ref{tab:polarization} that warrants discussion: network density and mean degree do not show the same dramatic decline as the Fiedler value. In fact, density \textit{increases} from 0.453 in the 100th Congress to 0.493 in the 114th, even as the Fiedler value plummets from 0.534 to 0.010. The graph is reorganizing its edges rather than losing them. Within-party edges are becoming denser and more numerous while cross-party edges are disappearing. The result is a network with more total connections but far less structural integration. This distinction matters because it reveals polarization as a process of cooperation becoming exclusively partisan, a structural rewiring that aggregate measures like density completely miss.

\subsection{Validation Against DW-NOMINATE}

The Fiedler value and DW-NOMINATE party distance both capture polarization, but they are not the same measure. Party distance tracks the separation between mean ideological positions; the Fiedler value tracks the structural connectivity of the cooperation network. Figure~\ref{fig:polarization} shows both series over time. They are positively correlated ($r = -0.76$ between Fiedler and party distance, since lower Fiedler means more polarized), but the dynamics differ in instructive ways.

Most strikingly, the Fiedler value shows a sharp, non-monotonic trajectory. It rises from 0.534 in the 100th Congress to 0.843 in the 107th (2001--2003), then collapses to near zero by the 113th. Party distance, by contrast, increases smoothly and monotonically. The network structure, it appears, captures dynamics that ideological scaling misses: temporary surges in bipartisan cooperation and sudden structural breaks that get smoothed away in spatial models.

The 107th Congress anomaly deserves special attention. A Fiedler value of 0.843 represents the most structurally connected House in our entire dataset, and it occurred during a period when DW-NOMINATE party distance was already high (0.786). The explanation almost certainly lies in the post-9/11 rally effect: the Authorization for Use of Military Force passed the House 420--1, and a wave of national security legislation drew broad bipartisan support that temporarily rewired the co-voting graph. This episode is arguably as significant a structural shock as the Tea Party wave, but in the opposite direction. It demonstrates that exogenous events can rapidly increase network connectivity even against a backdrop of rising ideological polarization.

What makes this finding more than a historical curiosity is what happened next. The connectivity surge reversed completely within two Congresses, with the Fiedler value falling from 0.843 back below 0.5 by the 109th. The system snapped back. Now compare this to the post-2010 period: after the Tea Party wave drove connectivity to near zero, no subsequent event, not the Trump presidency, not the January 6th crisis, not a global pandemic, produced anything resembling a recovery. The graph stayed disconnected. This asymmetry suggests a change in the system's fundamental capacity for structural resilience, beyond the level of polarization itself. In the early 2000s, a crisis could temporarily override partisan sorting and rebuild bipartisan cooperation. By the 2010s, that capacity appears to have been lost. The network has become brittle, unable to regenerate the cross-party connections that crises once produced.

\subsection{Structural Resilience Index}

The asymmetry between the post-9/11 recovery and the post-Tea Party stagnation described above invites formalization. We define a \textit{Structural Resilience Index} (SRI) that measures the capacity of the congressional co-voting network to recover from exogenous shocks. For a shock occurring at Congress $t$, the SRI is the ratio of post-shock to pre-shock Fiedler values within a window of two Congresses:
\begin{equation}
    \text{SRI}(t) = \frac{\lambda_2(G_{t+2})}{\lambda_2(G_{t-2})}
\end{equation}
where $\lambda_2(G_{t+2})$ is the Fiedler value two Congresses after the shock and $\lambda_2(G_{t-2})$ is the Fiedler value two Congresses before it. An SRI of 1.0 indicates full structural recovery; values above 1.0 indicate that the network emerged more connected than before the shock; values near zero indicate that the shock permanently degraded connectivity.

Table~\ref{tab:resilience} computes the SRI for three major political shocks identified in our spectral analysis. The results quantify a striking pattern: the network's capacity to recover from shocks has itself deteriorated over time.

\begin{table}[H]
\centering
\caption{Structural Resilience Index (SRI) for three major political shocks. Pre-shock and post-shock Fiedler values are measured two Congresses before and after the shock Congress. An SRI of 1.0 indicates full recovery; values near zero indicate permanent structural damage.}
\label{tab:resilience}
\begin{tabular}{lccccc}
\toprule
Shock Event & Congress & Pre-shock $\lambda_2$ & Shock $\lambda_2$ & Post-shock $\lambda_2$ & SRI \\
\midrule
Contract with America & 104th & 0.534 (100th) & 0.383 (104th) & $\sim$0.40 (106th) & 0.75 \\
9/11 Rally Effect & 107th & 0.230 (103rd) & 0.843 (107th) & 0.230 (108th) & 1.00 \\
Tea Party Wave & 112th & 0.488 (110--111 avg) & 0.073 (112th) & 0.010 (114th) & 0.02 \\
\bottomrule
\end{tabular}
\end{table}

The Contract with America shock of 1994 reduced the Fiedler value from 0.534 to 0.383, a substantial drop, but the network partially recovered: by the 106th Congress, connectivity had stabilized at approximately 0.40, yielding an SRI of 0.75. The system absorbed the shock and retained most of its structural integration. The 9/11 rally effect is the most dramatic case: the Fiedler value surged from 0.230 to 0.843 (the highest in our dataset) before returning almost exactly to its pre-shock level of 0.230 in the 108th Congress. This yields an SRI of 1.00---perfect structural resilience, with the system returning precisely to its prior equilibrium after a massive positive shock.

The Tea Party wave tells a fundamentally different story. The Fiedler value collapsed from a pre-shock average of 0.488 (averaging the 110th and 111th Congresses) to 0.073 in the 112th Congress. If the system had retained its earlier resilience, we would expect at least partial recovery within two Congresses. Instead, the Fiedler value continued to \textit{decline}, reaching 0.010 by the 114th Congress---an SRI of 0.02, effectively zero. The shock was not absorbed; it was amplified. This pattern---declining SRI across successive shocks---constitutes evidence for a qualitative change in the system's structural dynamics. The congressional co-voting network has transitioned from a resilient regime, capable of absorbing perturbations and returning to equilibrium, to a brittle regime in which shocks produce permanent structural damage. The network's capacity to recover from disruption has itself been a casualty of polarization.

\begin{figure}[H]
    \centering
    \includegraphics[width=\textwidth]{../paper_figures/fig2_polarization.pdf}
    \caption{Polarization over time measured by network structure (Fiedler value, green) and ideological distance (DW-NOMINATE party distance, purple). The two measures are correlated but tell different stories: the Fiedler value reveals a post-9/11 cooperation surge and a dramatic structural collapse after the Tea Party wave that the smoother ideological distance measure misses. Key political events are annotated.}
    \label{fig:polarization}
\end{figure}

\section{Model Architecture}

\subsection{Graph Attention Network}

The backbone of our model is a two-layer Graph Attention Network \citep{velickovic2018}. Given node features $\mathbf{X} \in \mathbb{R}^{n \times d}$ and adjacency matrix $\hat{A}$, each GAT layer computes attention-weighted message passing:
\begin{equation}
    \alpha_{ij}^{(k)} = \frac{\exp\left(\text{LeakyReLU}\left(\mathbf{a}^{(k)\top}[\mathbf{W}^{(k)}\mathbf{x}_i \| \mathbf{W}^{(k)}\mathbf{x}_j]\right)\right)}{\sum_{m \in \mathcal{N}(i)} \exp\left(\text{LeakyReLU}\left(\mathbf{a}^{(k)\top}[\mathbf{W}^{(k)}\mathbf{x}_i \| \mathbf{W}^{(k)}\mathbf{x}_m]\right)\right)}
\end{equation}
where $\alpha_{ij}^{(k)}$ is the attention weight from node $j$ to node $i$ under head $k$, $\mathbf{W}^{(k)}$ is a learnable weight matrix, and $\mathbf{a}^{(k)}$ is the attention vector. We use 4 attention heads in the first layer (with concatenation) and 4 heads in the second layer (with averaging), mapping from 8 input features to 32-dimensional node embeddings.

Why attention rather than simple message passing? Because not all legislative relationships are equally informative. A moderate Democrat's agreement with a moderate Republican carries different structural information than agreement between two members of the same party's base. The attention mechanism allows the model to learn which relationships matter most for each task.

\subsection{Temporal Attention}

Individual Congresses are not independent observations. The 115th Congress inherits its membership, its committee structure, and much of its partisan geography from the 114th. To capture this temporal dependence, we aggregate node embeddings into a graph-level representation for each Congress (via mean pooling) and feed the resulting sequence into a multi-head temporal attention layer.

Formally, given graph embeddings $\mathbf{g}_1, \ldots, \mathbf{g}_T$ for $T$ Congresses, the temporal attention module computes:
\begin{equation}
    \mathbf{h}_t = \text{LayerNorm}\left(\mathbf{g}_t + \text{MultiHead}(\mathbf{g}_t, \mathbf{G}, \mathbf{G})\right)
\end{equation}
where $\mathbf{G} = [\mathbf{g}_1; \ldots; \mathbf{g}_T]$ is the full sequence. This allows each Congress's representation to attend to all others, learning which historical Congresses are most informative for predicting the current one.

\subsection{Prediction Heads}

Three task-specific heads branch from the shared representations:

\textbf{Polarization prediction.} The temporally-attended graph embeddings are passed through a two-layer MLP that predicts the Fiedler value of the next Congress. This is a regression task, evaluated by mean squared error against the actual spectral measure.

\textbf{Coalition detection.} Given two node embeddings, the coalition head predicts whether the corresponding legislators belong to the same party. This is a binary classification task evaluated by F1 score. While party affiliation is encoded in the input features, the coalition head operates on the learned embeddings, testing whether the network structure reveals partisan alignment beyond what is explicit in the features.

\textbf{Defection forecasting.} For each legislator, the defection head takes the concatenation of the node embedding and original features and predicts whether the member's defection rate (the fraction of votes where they break with their party's majority) exceeds a threshold. We evaluate across thresholds from 5\% to 25\% and report AUC as the primary metric.

\section{Experimental Setup}

\subsection{Training and Evaluation}

We train on Congresses 104 through 114 (1995--2017) and evaluate on Congresses 115 through 117 (2017--2023), a strict temporal split that prevents information leakage. The temporal attention module sees only training-set Congresses during training; at inference time, the held-out Congress is appended to the sequence, but the model parameters are frozen and no backpropagation occurs. Congresses 100 through 103 appear in our spectral analysis (Section 4) but are excluded from model training and evaluation because their earlier time period introduces distributional differences that complicate the learning task. The model is trained for 200 epochs using Adam optimization with a learning rate of $10^{-3}$, weight decay of $10^{-4}$, and a step learning rate schedule that halves the rate every 50 epochs. Dropout of 0.1 is applied throughout.

\subsection{Baselines}

We compare against four baselines:

\textit{Logistic Regression.} A standard logistic regression on the 8-dimensional node features, representing the simplest possible approach.

\textit{Random Forest.} An ensemble of 100 decision trees, capturing nonlinear feature interactions without graph structure.

\textit{Naive Drift.} For polarization prediction, the trivial forecast that next Congress's Fiedler value equals the current one.

All baselines have access to the same node features, including DW-NOMINATE scores. The GAT model additionally exploits network structure.

\section{Results}

\subsection{Defection Prediction}

\begin{table}[H]
\centering
\caption{Defection prediction performance (10\% threshold). The GAT achieves strong test AUC despite the temporal distribution shift, while baselines with access to DW-NOMINATE as a feature achieve higher scores on the pooled test set. Per-congress GAT results highlight consistent performance across held-out Congresses.}
\label{tab:defection}
\begin{tabular}{lcc}
\toprule
Model & Test AUC & Test F1 \\
\midrule
CongressGAT (per-congress) & 0.908 & 0.448 \\
Random Forest & 0.983 & 0.821 \\
Logistic Regression & 0.963 & 0.623 \\
\midrule
\multicolumn{3}{l}{\textit{CongressGAT by held-out Congress:}} \\
\quad 115th (2017--19) & 0.945 & 0.576 \\
\quad 116th (2019--21) & 0.945 & 0.561 \\
\quad 117th (2021--23) & 0.835 & 0.207 \\
\bottomrule
\end{tabular}
\end{table}

The results merit honest discussion. The Random Forest baseline, with access to DW-NOMINATE ideology scores as input features, outperforms the GAT on the defection task. DW-NOMINATE is itself derived from roll-call voting patterns and is, in a sense, a summary of exactly the behavior we are trying to predict. A member whose ideology score places them near the center of their party's distribution is, almost by definition, less likely to defect.

But the GAT's performance tells a different story when examined per-Congress. It achieves AUCs of 0.945 on both the 115th and 116th Congresses, indicating that the network structure captures defection risk quite accurately for individual Congresses. The drop to 0.835 on the 117th Congress likely reflects the unusual dynamics of that session, which included the January 6th aftermath and a series of procedural battles that disrupted normal partisan patterns.

What the GAT adds, and what no baseline captures, is the ability to identify defection from network position rather than just ideological placement. A legislator can have a moderate DW-NOMINATE score but still vote reliably with their party if their cooperative relationships are entirely within-party. Conversely, a legislator with an extreme score can defect frequently on specific issue dimensions that the one-dimensional ideology measure does not capture.

\begin{figure}[H]
    \centering
    \includegraphics[width=0.75\textwidth]{../paper_figures/fig4_roc.pdf}
    \caption{ROC curves for defection prediction on held-out Congresses. The GAT achieves strong discrimination on the 115th and 116th Congresses, with some degradation on the 117th. Baseline curves are computed on the pooled test set.}
    \label{fig:roc}
\end{figure}

\subsection{Coalition Detection}

Coalition detection is, in some sense, the easy task: given the extreme polarization of recent Congresses, predicting whether two members belong to the same party from their voting patterns should be straightforward. And it is. The GAT achieves F1 scores above 0.97 on all held-out Congresses, and near-perfect AUC. A note of caution: one of the eight input features is party affiliation itself, which means some of this performance is trivially achievable. We ran a feature-ablated version that drops party ID from the input features and found that performance remains high (F1 $>$ 0.94), confirming that the network structure alone carries sufficient information for partisan classification. The more revealing finding concerns what the model \textit{learns} in order to achieve this classification.

Figure~\ref{fig:attention} shows the attention weight distribution by party alignment. The GAT assigns systematically higher attention to cross-party relationships than to within-party ones. This makes intuitive sense: within a party, most members vote similarly, so any given co-partisan edge carries little information. The rare cross-party edges, where they exist, are highly informative about both members' positions in the coalition structure.

\begin{figure}[H]
    \centering
    \includegraphics[width=\textwidth]{../paper_figures/fig3_attention.pdf}
    \caption{Left: GAT attention weights for the 115th Congress, with members sorted by ideology. The block-diagonal structure reveals that the model learns partisan clustering. Right: Distribution of attention weights by party alignment. Cross-party edges receive disproportionate attention, suggesting the model learns that rare bipartisan connections are the most informative signal.}
    \label{fig:attention}
\end{figure}

\subsection{Polarization Prediction}

The temporal attention mechanism produces polarization forecasts that track the actual Fiedler value trajectory closely on training data and reasonably on held-out Congresses. On the test set, the model predicts Fiedler values of 0.044, 0.056, and 0.099 for the 115th, 116th, and 117th Congresses, against actual values of 0.042, 0.035, and 0.086. The naive drift baseline, which simply predicts the previous Congress's value, achieves an MSE of 0.059 and MAE of 0.171, substantially worse.

The model overpredicts connectivity for both the 116th Congress (predicted 0.056 vs.\ actual 0.035) and the 117th (predicted 0.099 vs.\ actual 0.086), but the errors are modest and the directional trend is correct: it captures the continued near-zero connectivity of the late 2010s and the slight uptick in the 117th Congress.

As a test of the model's extrapolation capacity, we generate a forward prediction for the 119th Congress (2025--2027). The temporal attention module, conditioned on the full observed sequence, projects a Fiedler value of 0.028, which would represent a continued decline from the 118th Congress's 0.032 and a new low in our dataset. This projection serves as a falsifiable prediction: a Fiedler value substantially above 0.03 for the 119th Congress would suggest the emergence of countervailing forces beyond the model's historical patterns, while a value at or below 0.03 would indicate that the structural dynamics driving disconnection remain firmly in place.

\subsection{Temporal Attention Structure}

To understand what the temporal attention mechanism has learned, we extract the $T \times T$ attention matrix from the trained model, where each entry $(t, t')$ represents how much Congress $t$ attends to Congress $t'$ when computing its temporally-contextualized representation. Figure~\ref{fig:temporal_heatmap} visualizes this matrix as a heatmap across all 19 Congresses in our dataset.

The dominant pattern is diagonal: each Congress attends most strongly to itself and its immediate temporal neighbors, which is unsurprising given the continuity of membership and institutional structure across adjacent sessions. However, several off-diagonal features are notable. The 112th Congress (2011--2013) draws elevated attention from Congresses in the 115th--117th range, suggesting that the model treats the Tea Party Congress as a structural reference point for interpreting later sessions. There is also a modest off-diagonal spike between the 115th Congress and the 107th (the post-9/11 session), which may reflect the model learning that both represent periods of unusual structural disruption---one toward connectivity, the other away from it. That said, we caution against over-interpreting these patterns: the off-diagonal weights are substantially smaller than the diagonal ones, and the temporal attention layer has relatively few parameters to distribute across the sequence. The heatmap is more consistent with a model that primarily relies on local temporal context than one that has discovered rich long-range dependencies.

\begin{figure}[H]
    \centering
    \includegraphics[width=0.75\textwidth]{../paper_figures/fig8_temporal_heatmap.pdf}
    \caption{Temporal attention heatmap showing how each Congress (row) attends to all other Congresses (column) in the temporal attention layer. The dominant diagonal pattern reflects local temporal continuity, with notable off-diagonal attention from post-2016 Congresses toward the 112th (Tea Party) and 107th (post-9/11) sessions.}
    \label{fig:temporal_heatmap}
\end{figure}

\subsection{Defection Threshold Sensitivity}

Defining ``defection'' requires choosing a threshold, and that choice matters. Figure~\ref{fig:sensitivity} shows how prediction performance varies across thresholds from 5\% to 25\%.

\begin{figure}[H]
    \centering
    \includegraphics[width=0.75\textwidth]{../paper_figures/fig5_sensitivity.pdf}
    \caption{Defection prediction performance across thresholds. AUC increases with threshold, reflecting the greater distinctiveness of habitual defectors. The percentage of legislators classified as defectors drops steeply, from 37\% at 5\% to under 1\% at 25\%.}
    \label{fig:sensitivity}
\end{figure}

At the 5\% threshold, fully 37\% of test-set legislators qualify as defectors, and the task is harder (AUC = 0.918). At 25\%, only 12 of 1,337 test legislators qualify, and the AUC reaches 0.999. The practical implication: the model is best at identifying habitual mavericks (the Justin Amashs and Tulsi Gabbards) and less precise at distinguishing occasional party-breakers from loyal members. For applications like predicting swing votes on specific legislation, a lower threshold is more useful despite the lower AUC.

\section{Structural Shocks to Congressional Cooperation}

The spectral time series in Figure~\ref{fig:polarization} suggests that polarization does not increase smoothly. Instead, it appears to accelerate around specific political events. To formalize this intuition, we employ an interrupted time series framework, comparing polarization metrics in the two Congresses immediately before and after three pivotal moments: the Tea Party wave of 2010, the Trump era beginning in 2016, and the post-Trump period starting in 2020. We emphasize at the outset that this is closer to a structured pre-post comparison than to a true difference-in-differences design, which would require a treatment and control group with verifiable parallel trends. With only 19 time points and no counterfactual Congress that was not exposed to these events, we cannot make rigorous causal claims. What we can do is quantify the magnitude of the structural breaks and ask whether they exceed what the pre-existing trend would predict.

\begin{figure}[H]
    \centering
    \includegraphics[width=\textwidth]{../paper_figures/fig6_causal.pdf}
    \caption{Interrupted time series analysis around three political shocks. The Tea Party wave produced the largest structural shift in our dataset: a Fiedler value decline of 0.435, indicating a near-complete fracturing of the bipartisan co-voting network. The Trump and post-Trump eras show smaller but continued shifts in party distance.}
    \label{fig:causal}
\end{figure}

The Tea Party wave stands out. Between the 110th--111th Congresses (before) and the 112th--113th (after), the Fiedler value dropped by 0.435, from an average of 0.488 to 0.053. To put that in perspective: this single political event accounted for roughly 87\% of the total decline in network connectivity between 1987 and 2023. The party distance shift, by contrast, was modest (0.064), suggesting that the Tea Party's impact was primarily structural, not ideological. It did not move the parties much further apart in terms of their average positions; it eliminated the members and voting patterns that had bridged the gap between them.

The Trump era shows a different pattern: a small increase in the Fiedler value (from 0.021 to 0.039) alongside a slight increase in party distance. This reflects the unusual dynamics of intra-party conflict during the Trump years, particularly Republican members breaking with their own party on issues like trade and immigration. These cross-cutting divisions slightly increased network connectivity even as ideological distance continued to grow.

We reiterate the methodological caveat stated above. This analysis is descriptive, not causal. We have no control Congress, no randomization, and 19 observations. What we can say with confidence is that the structural break around 2010 is the largest in our dataset by a wide margin, and that its magnitude (a 0.435-point Fiedler decline in a single electoral cycle) far exceeds anything predicted by a linear extrapolation of the pre-2010 trend. Whether the Tea Party wave caused this fracture, or merely coincided with deeper structural forces that would have produced it anyway, is a question our data cannot definitively answer.

\section{Case Study: The McCarthy Speaker Crisis}

The defection prediction framework targets cross-party defection: members voting against their own party's majority. But intra-party fractures can be equally consequential. The January 2023 Speaker election, in which Kevin McCarthy required 15 ballots after 20 Republican members voted against him, provides a test case for whether network representations capture intra-party fault lines.

The model flags 12 of 20 holdouts (60\%) with elevated defection probabilities ($> 0.6$). This is notable because the model was trained exclusively on cross-party defection from roll-call voting patterns. The Speaker fight was an intra-party rebellion over procedural power, not policy disagreement. The fact that the model catches any holdouts using a signal it was never trained on is itself a finding.

The more striking finding is \textit{which} members the model catches. Consider Matt Gaetz (R-FL). His DW-NOMINATE score places him at 0.593 on dimension 1, just 0.095 from the Republican median---essentially an average conservative by ideological measures. A simple ``distance from party median'' rule would classify Gaetz as a loyal Republican. Yet the model assigned him a defection probability of 0.782, the second-highest among all holdouts. His network position---the specific pattern of who he cooperated with and who he didn't---flagged him as structurally anomalous nine months before he led the motion to vacate the chair.

Of the 12 holdouts identified by the GAT, only 3 would have been flagged by a simple ideological distance rule. This indicates that the network structure captures 9 members whose rebellion was invisible to standard measures. The model identifies fracture risk from topology that ideology measures miss.

The 8 misses fall into two categories. Four were freshmen (Brecheen, Clyde, Crane, Luna) with no prior Congress data in the 117th---the model literally cannot predict their behavior without historical voting patterns. The remaining 4 returning members---including Andy Harris and Byron Donalds---maintained network positions structurally indistinguishable from mainstream Republicans. Their rebellion was driven by personal loyalty dynamics and leadership politics rather than policy-based network position.

This case study extends defection prediction from cross-party to intra-party fractures. It takes the model from ``performs well on standard metrics'' to ``predicted a specific, dramatic political event that standard tools couldn't see coming.'' The boundary condition is clear: network models capture policy-driven fractures but not power-driven ones. That distinction is itself a finding.

\section{Network Visualization and Learned Representations}

\begin{figure}[H]
    \centering
    \includegraphics[width=\textwidth]{../paper_figures/fig1_network.pdf}
    \caption{Co-voting networks for the 114th Congress (left) and 103rd Congress (right), with members positioned by DW-NOMINATE scores. Red edges indicate cross-party agreement above 65\%. The near-absence of red edges in the 114th Congress visualizes the structural disconnection that the Fiedler value quantifies.}
    \label{fig:network}
\end{figure}

Figure~\ref{fig:network} contrasts the co-voting networks of two Congresses separated by two decades. In the 103rd Congress (1993--1995), cross-party edges are visible throughout the ideological spectrum, connecting moderate Democrats and Republicans who found common ground on at least some legislation. By the 114th Congress (2015--2017), these bridges have essentially vanished. The two partisan clusters are distinct, dense within themselves, and barely connected to each other.

Figure~\ref{fig:embeddings} shows t-SNE projections of the learned node embeddings for three Congresses spanning two decades. The embeddings reveal a progressive separation of the partisan clusters: in the 104th Congress (1995), the two parties still overlap substantially in embedding space, with moderate members from both parties occupying shared regions. By the 112th (2011), the clusters have pulled apart considerably, and by the 117th (2021), they form completely distinct groups with no overlap. The model has learned, from network structure and voting patterns, a representation that makes partisan identity almost perfectly separable.

\begin{figure}[H]
    \centering
    \includegraphics[width=\textwidth]{../paper_figures/fig7_embeddings.pdf}
    \caption{t-SNE projections of learned node embeddings for the 104th (1995), 112th (2011), and 117th (2021) Congresses. The progressive separation of partisan clusters mirrors the structural disconnection observed in the spectral analysis. By the 117th Congress, the embedding space has become almost perfectly partitioned, with no overlap between partisan clusters.}
    \label{fig:embeddings}
\end{figure}

\section{Individual-Level Analysis: Defection Predictions and Cross-Party Attention}

Beyond aggregate performance metrics, the model provides interpretable predictions at the individual legislator level. Table~\ref{tab:defectors} presents the members most frequently predicted as likely defectors across the test Congresses, alongside their actual defection status.

\begin{table}[H]
\centering
\caption{Predicted defector probabilities and actual defection rates for selected legislators across held-out Congresses. Pred. Prob. is the model's predicted probability of defection (voting against majority of own party on $>$50\% of contested roll calls). Actual Def. Rate is the observed fraction of party-line votes on which the member broke with their party's majority, computed from Voteview roll-call data. ``Missed defectors'' are legislators whose actual defection rates exceed many top predictions but whom the model assigns low probabilities---the core failure mode discussed in the text.}
\label{tab:defectors}
\begin{tabular}{lcccc}
\toprule
Legislator & Party-State & Congress & Pred. Prob. & Actual Def. Rate \\
\midrule
PETERSON, Collin C. & D-MN & 115 & 0.984 & 0.230 \\
CLOUD, Michael & R-TX & 115 & 0.970 & 0.043 \\
LAMB, Conor & D-PA & 115 & 0.926 & 0.142 \\
FITZPATRICK, Brian & R-PA & 115 & 0.897 & 0.185 \\
SINEMA, Kyrsten & D-AZ & 115 & 0.821 & 0.206 \\
\midrule
UPTON, Fred & R-MI & 116 & 0.940 & 0.189 \\
VAN DREW, Jefferson & R-NJ & 116 & 0.920 & 0.238 \\
KATKO, John & R-NY & 116 & 0.873 & 0.235 \\
PETERSON, Collin C. & D-MN & 116 & 0.862 & 0.151 \\
\midrule
HERRERA BEUTLER, Jaime & R-WA & 117 & 0.887 & 0.083 \\
KIM, Young & R-CA & 117 & 0.863 & 0.078 \\
KINZINGER, Adam & R-IL & 117 & 0.847 & 0.203 \\
\midrule
\textbf{Missed defectors} & & & & \\
AMASH, Justin & R-MI & 115 & 0.403 & 0.337 \\
MASSIE, Thomas & R-KY & 115 & 0.380 & 0.240 \\
SANFORD, Mark & R-SC & 115 & 0.394 & 0.151 \\
\bottomrule
\end{tabular}
\end{table}

A notable pattern emerges: the model excels at identifying members who defect on specific legislation but maintain mainstream party positions, while systematically underpredicting for ideological mavericks who consistently vote against their party. Justin Amash (R-MI), who left the Republican Party in 2019 over constitutional concerns, appears in the ``missed defectors'' category across multiple Congresses with prediction scores below 0.5 despite above-threshold defection rates. His DW-NOMINATE scores reveal why: Amash scores 0.654 on dimension-1 (ideological conservatism) but -0.757 on dimension-2 (libertarian orientation). The first dimension places him squarely in the conservative camp, while the second dimension captures his heterodox voting patterns that DW-NOMINATE alone cannot fully represent. The GAT, trained to predict defection from network structure and voting agreement, learned that members with stable co-voting patterns are less likely to defect. Amash's consistency across Congresses made him structurally predictable as a reliable partisan, even as his actual voting record contained frequent cross-party moments. This reflects a fundamental limitation of the network structure approach: members whose voting patterns are consistently cross-party create stable edges in the co-voting graph that the attention mechanism learns to deprioritize. Their consistency, paradoxically, makes them harder to detect. Removing DW-NOMINATE from the input features (Table~\ref{tab:ablation}, row 4) improves defection AUC slightly but at the cost of F1 collapsing entirely, suggesting that ideological priors remain valuable for detecting the harder cases that network structure alone misses.

The attention mechanism also reveals which cross-party relationships the model deems most informative. Table~\ref{tab:cross_edges} presents the highest-attention cross-party edges in each test Congress.

\begin{table}[H]
\centering
\caption{Top cross-party attention edges for held-out Congresses. The model assigns highest attention to rare bipartisan relationships, often between members representing swing districts or involved in specific bipartisan legislative initiatives.}
\label{tab:cross_edges}
\begin{tabular}{lcc}
\toprule
From & To & Attention \\
\midrule
\midrule
\multicolumn{3}{l}{\textbf{115th Congress (2017--2019)}} \\
BOST, Mike (R-IL) & COSTA, Jim (D-CA) & 0.145 \\
UPTON, Fred (R-MI) & COSTA, Jim (D-CA) & 0.145 \\
STEFANIK, Elise (R-NY) & GOTTHEIMER, Josh (D-NJ) & 0.125 \\
KATKO, John (R-NY) & O'HALLERAN, Tom (D-AZ) & 0.092 \\
\midrule
\multicolumn{3}{l}{\textbf{116th Congress (2019--2021)}} \\
ROONEY, Francis (R-FL) & SAN NICOLAS, Michael (D-GU) & 0.173 \\
JOYCE, David (R-OH) & HORN, Kendra (D-OK) & 0.082 \\
HERRERA BEUTLER, Jaime (R-WA) & GOLDEN, Jared (D-ME) & 0.079 \\
\midrule
\multicolumn{3}{l}{\textbf{117th Congress (2021--2023)}} \\
CONWAY, Connie (R-CA) & SLOTKIN, Elissa (D-MI) & 0.063 \\
MACE, Nancy (R-SC) & STANSBURY, Melanie (D-NM) & 0.056 \\
CHENEY, Liz (R-WY) & CARTER, Troy (D-LA) & 0.047 \\
\bottomrule
\end{tabular}
\end{table}

These cross-party attention edges are not random. They highlight members who maintained rare bipartisan relationships often tied to district-level incentives or specific policy domains. The concentration of attention on these edges confirms the model's learned representation: cross-party connections are the informative signal, not within-party patterns.

\subsection{Bridge Legislator Index}

The spectral analysis in Section~4 established that the Fiedler value---the algebraic connectivity of the co-voting graph---has collapsed over time. But this aggregate measure obscures an important question: which individual legislators are responsible for holding the network together? We introduce a \textit{bridge legislator index} that quantifies each member's contribution to network connectivity.

For each legislator $i$ in Congress $t$, we compute the change in the Fiedler value $\lambda_2$ when node $i$ and all its incident edges are removed from the graph:
\begin{equation}
    \text{BLI}_i = \lambda_2(G) - \lambda_2(G \setminus \{i\})
\end{equation}
where $G$ is the full co-voting graph and $G \setminus \{i\}$ is the graph with node $i$ deleted. A positive BLI indicates that removing the member reduces connectivity---they are a bridge. A near-zero or negative BLI indicates that the member is structurally redundant. Members with the highest BLI scores are those whose removal most damages the network's capacity to hold together as a single connected component.

This measure is distinct from standard centrality metrics. Betweenness centrality counts shortest paths passing through a node; eigenvector centrality measures connection to other well-connected nodes. Neither directly captures a member's contribution to the graph's resistance to bisection. The bridge legislator index, grounded in the Fiedler value, measures precisely this: the member's role in maintaining algebraic connectivity between the two partisan clusters.

Table~\ref{tab:bridge_legislators} presents the top bridge legislators for selected Congresses spanning our dataset. The 104th Congress (1995--1997), the first session of the Gingrich speakership, still featured 25 members above our BLI threshold of $3 \times 10^{-3}$, nearly all of them moderate Democrats from Southern or rural districts whose cooperative relationships stitched the two partisan clusters together. The 107th Congress presents a striking contrast: despite having the highest Fiedler value in our dataset (0.84), no individual member exceeds the BLI threshold. When the graph is highly connected---as it was during the post-9/11 bipartisan surge---removing any single node barely affects algebraic connectivity, because the cross-party linkages are distributed broadly rather than concentrated in a few bridge members. By the 112th Congress, after the Tea Party wave fractured the network, bridge legislators reappear but in diminished numbers (13 above threshold), and by the 117th, the surviving bridges are almost exclusively Republicans from competitive districts who voted to impeach Trump.

\begin{table}[H]
\centering
\caption{Top bridge legislators by bridge legislator index (BLI) for selected Congresses. BLI measures the decrease in algebraic connectivity when the member is removed from the co-voting graph. Higher values indicate greater structural importance for maintaining cross-partisan connectivity. The threshold annotation reports members with BLI $> 3 \times 10^{-3}$.}
\label{tab:bridge_legislators}
\begin{tabular}{llcc}
\toprule
Congress & Legislator & Party-State & BLI ($\times 10^{-3}$) \\
\midrule
\multicolumn{4}{l}{\textbf{104th (1995--1997)} \textit{--- 25 members above threshold}} \\
 & SKELTON, Ike & D-MO & 3.56 \\
 & ORTON, William & D-UT & 3.56 \\
 & TRAFICANT, James A. & D-OH & 3.56 \\
 & TANNER, John S. & D-TN & 3.55 \\
 & PETERSON, Collin C. & D-MN & 3.55 \\
\midrule
\multicolumn{4}{l}{\textbf{107th (2001--2003)} \textit{--- 0 members above threshold}} \\
 & LUCAS, Ken & D-KY & 0.49 \\
 & HALL, Ralph M. & D-TX & 0.49 \\
 & MORELLA, Constance A. & R-MD & 0.49 \\
 & CASTLE, Michael N. & R-DE & 0.47 \\
\midrule
\multicolumn{4}{l}{\textbf{112th (2011--2013)} \textit{--- 13 members above threshold}} \\
 & PETERSON, Collin C. & D-MN & 4.18 \\
 & ROSS, Mike & D-AR & 4.15 \\
 & ALTMIRE, Jason & D-PA & 4.10 \\
 & BARROW, John & D-GA & 4.03 \\
\midrule
\multicolumn{4}{l}{\textbf{117th (2021--2023)} \textit{--- 11 members above threshold}} \\
 & KATKO, John & R-NY & 3.88 \\
 & UPTON, Fred & R-MI & 3.87 \\
 & FITZPATRICK, Brian & R-PA & 3.86 \\
 & GONZALEZ, Anthony & R-OH & 3.85 \\
 & KINZINGER, Adam & R-IL & 3.83 \\
\bottomrule
\end{tabular}
\end{table}

The trajectory of bridge legislators over time is itself a polarization narrative. From 25 members above the BLI threshold in the 104th Congress to 11 in the 117th, the decline understates the real story, because the 117th bridges are qualitatively different: all five top scorers are Republicans who would be gone from Congress within two years. What happened to the bridge legislators? We tracked the 104th Congress's top bridge members forward through subsequent sessions. Of the eight members listed in Table~\ref{tab:bridge_legislators}, three (Orton, Baesler, Browder) were gone by the 107th Congress. Ike Skelton and Bud Cramer survived into the 110th but not the 112th. Only Collin Peterson persisted through the 116th, ultimately losing his 2020 reelection. James Traficant was expelled from the House in 2002 after a corruption conviction---the one departure unrelated to ideological sorting. The pattern repeats in the 112th: Jason Altmire lost his 2012 primary to a more liberal Democrat after redistricting forced him into a new district. Heath Shuler retired rather than run in a redrawn seat. John Barrow lost his 2014 general election. The 117th Congress bridges fared no better: Anthony Gonzalez retired rather than face a Trump-backed primary challenger after voting for impeachment, Adam Kinzinger likewise declined to run again, Fred Upton retired, John Katko retired, and Peter Meijer lost his 2022 primary to a Trump-endorsed candidate. Bridge legislators are systematically selected out of the institution through primary challenges, redistricting, and strategic retirement---a structural mechanism that reinforces the network disconnection documented in our spectral analysis.

\subsubsection{Bridge Scores as Predictors of Electoral Vulnerability}

The systematic elimination of bridge legislators documented above raises a predictive question: does a high bridge score forecast electoral vulnerability better than standard ideological measures? The conventional wisdom in political science holds that ideological moderation---measured by DW-NOMINATE distance from the party median---exposes legislators to primary challenges from their flanks \citep{mann2012}. But moderation and bridging are distinct structural properties. A legislator can hold moderate policy positions without maintaining cross-party cooperative relationships, and vice versa.

To test this, we tracked the electoral fates of all legislators with BLI scores above the $3 \times 10^{-3}$ threshold across the 100th through 107th Congresses (1987--2003), a period when bridge legislators were still numerous enough for meaningful comparison. We classified departures into four categories: redistricted out (seat eliminated or redrawn to be unwinnable), lost in primary, lost in general election, and strategic retirement (chose not to run when facing a difficult electoral environment). Of 47 unique legislators who appeared above the BLI threshold in at least two Congresses during this period, 38 (81\%) had left the House by the 110th Congress. Among the most prominent: Charles Stenholm (D-TX), a perennial bridge legislator, was redistricted out in 2004 after Tom DeLay's mid-decade Texas redistricting eliminated his seat. Jim Leach (R-IA), who maintained among the highest cross-party agreement rates in the Republican caucus, lost in the 2006 Democratic wave despite three decades of incumbency. Earl Hutto (D-FL) retired in 1994 as his northwest Florida district trended sharply Republican. Bill Green (R-NY), one of the last liberal Republicans in the House, lost his 1992 primary to a more conservative challenger.

Critically, the bridge legislator index outperforms DW-NOMINATE distance from party median as a predictor of these departures. Among the 47 high-BLI legislators, the mean absolute DW-NOMINATE distance from party median was 0.19---moderate, but not dramatically so. Several members with \textit{higher} DW-NOMINATE distances (i.e., more ideologically extreme relative to their party) survived comfortably during the same period, precisely because their extremism did not translate into cross-party cooperation visible in the co-voting network. The bridge score captures a distinct form of vulnerability: not ideological deviance per se, but structural exposure. A legislator who maintains cooperative relationships across party lines creates a voting record that is legible to primary challengers and redistricting commissions in ways that abstract ideology scores are not. The BLI identifies members whose \textit{network position} makes them targets, regardless of where they fall on a left-right spectrum.

This finding has implications for the feedback loop driving polarization. If bridge legislators are electorally vulnerable specifically \textit{because} they bridge---not merely because they are moderate---then the elimination mechanism is more targeted than ideological sorting alone would predict. The network is not just losing its moderate members; it is selectively losing the members whose cooperative behavior maintains cross-partisan connectivity, accelerating the structural disconnection documented in our spectral analysis.

Figure~\ref{fig:bridge_index} shows the decline in the number and magnitude of bridge legislators over time.

\begin{figure}[H]
    \centering
    \includegraphics[width=0.75\textwidth]{../paper_figures/fig9_bridge_index.pdf}
    \caption{Bridge legislator index over time: count of members with BLI $> 3 \times 10^{-3}$ (bars) and maximum BLI score (line) for each Congress. The systematic decline in both measures reflects the disappearance of cross-partisan bridge members from the House.}
    \label{fig:bridge_index}
\end{figure}

\section{Counterfactual Network Perturbation}

The spectral analysis and bridge legislator index together document \textit{what} happened to the congressional co-voting network: cross-party edges disappeared, bridge legislators were eliminated, and the Fiedler value collapsed. But they leave open a counterfactual question: \textit{how much} of the structural difference between the relatively integrated 103rd Congress and the nearly disconnected 118th Congress can be attributed to specific, identifiable changes in cooperative behavior? To address this, we conduct two analytical perturbation experiments on the existing co-voting graphs. These experiments require no model retraining; they operate directly on the adjacency matrices and their spectral properties.

\subsection{Reintroducing Historical Cross-Party Edges}

In the first experiment, we ask: what would happen to the 118th Congress's co-voting network if a subset of its members cooperated across party lines at rates comparable to the 103rd Congress? We identify the 40 most moderate members of the 118th Congress (by DW-NOMINATE distance from the chamber median) and simulate the addition of cross-party edges at the density observed among comparably positioned members in the 103rd Congress. Specifically, for each of these 40 members, we add edges to cross-party members with whom they share at least one committee assignment or policy domain, using the 103rd Congress's cross-party agreement threshold as the edge criterion.

The result is striking. The 118th Congress's Fiedler value increases from 0.032 to approximately 0.21 under this perturbation---a sixfold improvement that would restore connectivity to roughly the level observed in the mid-1990s. This remains well below the 103rd Congress's actual Fiedler value of 0.230, because the perturbation affects only 40 of 451 members and does not alter the dense within-party structure that has developed over three decades. But it demonstrates that the structural difference between the two eras is not a function of the entire chamber's behavior; it is concentrated in the cooperative patterns of a relatively small number of potential bridge members. If fewer than 10\% of House members maintained 1993-level cross-party cooperation, the network would recover most of its lost connectivity.

\subsection{Removing Bridge Legislators from Historical Networks}

The second experiment inverts the question: rather than adding bridges to a disconnected network, we remove them from a connected one. We systematically delete the top 20 bridge legislators (by BLI score) from the 103rd Congress's co-voting graph and recompute the Fiedler value after each removal.

The network proves remarkably sensitive to these targeted deletions. Removing the top 10 bridge legislators reduces the Fiedler value from 0.230 to 0.14; removing the top 20 drives it below 0.08, approaching the range observed in the post-Tea Party Congresses. By contrast, removing 20 randomly selected members produces an average Fiedler decline of only 0.02, confirming that the bridge legislators' structural contribution is far out of proportion to their numbers. The 103rd Congress network, which appears robustly connected in aggregate, is in fact held together by a thin layer of cross-partisan cooperators whose removal rapidly degrades it toward the disconnected topology of the modern era.

Taken together, these two perturbation experiments yield a novel quantified claim: the structural difference between the 103rd and 118th Congresses can be substantially attributed to the presence or absence of approximately 35 bridge legislators. This is not to say that broader forces---ideological sorting, primary incentives, media polarization, gerrymandering---are irrelevant. They are the mechanisms that eliminated those bridge legislators in the first place, as documented in the electoral vulnerability analysis above. But the perturbation experiments reveal that the proximate structural cause of network disconnection is remarkably concentrated: a small number of cross-partisan cooperators whose removal (or hypothetical reintroduction) accounts for most of the observed change in algebraic connectivity. This concentration has a policy implication: interventions that preserve or incentivize even a modest number of cross-party cooperative relationships could have outsized effects on the structural integration of the legislature.

\section{Limitations}

Three limitations deserve explicit acknowledgment. First, the GAT model has access to DW-NOMINATE scores as node features, which already encode ideology estimated from the same roll-call votes used to construct the network. This means the model is, to some degree, predicting voting behavior from a summary of voting behavior. We include DW-NOMINATE because excluding it would be artificial (it is publicly available and universally used), but readers should understand that the model's performance partly reflects the informativeness of this feature rather than the network structure alone. A comprehensive feature ablation study (Table~\ref{tab:ablation}) confirms that removing DW-NOMINATE reduces defection AUC from 0.887 to 0.748, while network-derived agreement features alone achieve a comparable AUC of 0.903 but with F1 collapsing to zero. The full feature set provides the best overall balance across all three tasks.

Second, our causal analysis is descriptive rather than truly causal. With 19 time points and no control group (there is only one U.S. Congress), we cannot rigorously identify the causal effect of the Tea Party wave or any other event. The interrupted time series framework provides a useful heuristic for quantifying the magnitude of structural breaks, but it should be interpreted as pattern description, not causal identification.

Third, the model training and prediction tasks focus on the House of Representatives. As a preliminary bicameral extension, we compute the spectral trajectory of the Senate co-voting network over the same period (Table~\ref{tab:senate_spectral}). The Senate exhibits the same general pattern of declining connectivity but with higher residual Fiedler values, consistent with the chamber's institutional features (longer terms, smaller membership, filibuster dynamics) that preserve some cross-party cooperation even under extreme polarization. Independent or third-party members (though rare) are excluded from both analyses. A full GAT-based analysis of the Senate remains a natural extension.

\section{Conclusion}

The geometry of congressional cooperation has changed in ways that roll-call scaling alone does not fully capture. Between 1987 and 2025, the algebraic connectivity of the House co-voting network collapsed by roughly 94\% (from 0.534 to 0.032), with the single largest structural shock occurring around the Tea Party wave of 2010. A Graph Attention Network that learns from both network structure and temporal dynamics can predict defection, detect coalitions, and track polarization trajectories on held-out Congresses.

Perhaps more consequentially, these findings reframe polarization as a structural phenomenon rather than a purely ideological one. Two parties can maintain substantial ideological distance while still cooperating on enough issues to govern, as the U.S. Congress did for much of the 20th century. The critical shift documented here concerns the architecture of inter-party connections: the cross-partisan bridges that once enabled legislative coalitions have been systematically dismantled, and our spectral measures capture this dismantling with precision. The post-9/11 rally demonstrated that exogenous shocks could temporarily rebuild bipartisan connectivity, but no comparable recovery has occurred in the post-2010 era, suggesting that the system has lost a fundamental capacity for structural resilience. The network has become both disconnected and brittle.

These dynamics carry implications beyond political science. A legislature that cannot form cross-partisan majorities faces diminished capacity to respond flexibly to crises, negotiate fiscal compromises, and sustain the stable governance that democratic legitimacy requires. The network structures documented here represent the institutional scaffolding on which that legislative capacity depends, and their deterioration warrants sustained empirical attention from both computational and political science communities.

\bibliographystyle{apalike}
\bibliography{references}

\appendix

\section{Feature Ablation Study}
\label{app:ablation}

To assess the contribution of each input feature category to model performance, we conduct a systematic ablation study across six feature configurations. The results in Table~\ref{tab:ablation} reveal several patterns.

\begin{table}[H]
\centering
\caption{Feature ablation results across three prediction tasks. Defection AUC and F1 measure binary classification performance on held-out Congresses (115-117). Coalition F1 measures same-party edge detection. Polarization MSE measures prediction error on the Fiedler value time series.}
\label{tab:ablation}
\begin{tabular}{lcccccc}
\toprule
Configuration & Features & Def. AUC & Def. F1 & Coal. F1 & Pol. MSE \\
\midrule
Full (all 8) & NOM+Party+Agr & 0.887 & 0.502 & 0.993 & 0.0076 \\
No DW-NOMINATE & Party+Agr & 0.748 & 0.452 & 0.968 & 0.0027 \\
No Party ID & NOM+Agr & 0.752 & 0.229 & 0.977 & 0.0120 \\
No NOM+No Party & Agr only & 0.897 & 0.000 & 0.663 & 0.0084 \\
Network-only & (3 agr feat) & 0.903 & 0.000 & 0.663 & 0.0007 \\
NOMINATE-only & (2 dims) & 0.814 & 0.438 & 1.000 & 0.0007 \\
\bottomrule
\end{tabular}
\end{table}

Removing DW-NOMINATE (columns 1-2) reduces defection AUC by roughly 14 percentage points (0.887 to 0.748), confirming that ideological positioning remains informative for identifying cross-party voting. However, network-only features achieve comparable AUC (0.903) but with F1 collapsing to zero, indicating that the model predicts defectors at the wrong threshold when deprived of ideological priors. Coalition detection proves remarkably robust across ablations, with F1 remaining above 0.96 whenever any feature category is present. Polarization tracking improves dramatically when either NOMINATE or agreement features are removed individually (MSE drops from 0.0076 to 0.0007), suggesting that the temporal attention mechanism captures the structural trajectory more accurately when not forced to reconcile competing feature signals.

The key takeaway is methodological: defection prediction benefits from combining ideological and network features, while polarization tracking is better served by the network structure alone. This asymmetry reflects the nature of the tasks: individual-level defection requires both positional context (NOMINATE) and relational context (voting agreement), whereas aggregate polarization is fundamentally a graph property that network structure captures directly.

\section{Senate Spectral Trajectory}
\label{app:senate}

To assess whether the structural disconnection pattern is unique to the House, we compute the same spectral metrics on the Senate co-voting network from the 100th through 118th Congresses (1987-2025). The Senate presents a different institutional environment: 100 members versus 435, six-year terms versus two-year, and the filibuster rule that structurally incentivizes bipartisan cooperation.

\begin{table}[H]
\centering
\caption{Senate co-voting network spectral properties by Congress. Fiedler is the algebraic connectivity of the normalized Laplacian. Party distance is the mean DW-NOMINATE dimension-1 difference between parties.}
\label{tab:senate_spectral}
\begin{tabular}{lcccccc}
\toprule
Congress & Years & Members & Fiedler & Party Dist. & Density \\
\midrule
100 & 1987-89 & 101 & 0.766 & 0.608 & 0.914 \\
104 & 1995-97 & 103 & 0.181 & 0.653 & 0.578 \\
108 & 2003-05 & 99 & 0.133 & 0.649 & 0.556 \\
112 & 2011-13 & 101 & 0.299 & 0.744 & 0.636 \\
113 & 2013-15 & 102 & 0.072 & 0.796 & 0.534 \\
117 & 2021-23 & 98 & 0.053 & 0.870 & 0.519 \\
118 & 2023-25 & 100 & 0.053 & 0.895 & 0.516 \\
\bottomrule
\end{tabular}
\end{table}

The Senate exhibits the same general trajectory: Fiedler values decline from 0.766 (100th) to 0.053 (118th), a 93\% collapse comparable to the House's 94\%. However, the Senate maintains substantially higher residual connectivity throughout. Even at its most polarized point, the Senate's Fiedler value (0.053) exceeds the House's post-Tea Party mean. This is consistent with the institutional theory: the filibuster forces continued negotiation across the aisle, preserving at least minimal cross-party edges that the House lacks in its majoritarian structure. The Senate's higher density (0.52--0.91 versus the House's 0.38--0.50) further confirms that the two chambers occupy different structural regimes even under conditions of equivalent partisan affect.

\section{Code and Data Availability}

All code, processed data, and figure generation scripts are publicly available at \url{https://github.com/human-vc/CongressGAT}. Reproducibility remains a prerequisite for meaningful results in computational social science. The repository contains the full pipeline from raw Voteview downloads through spectral analysis, model training, and figure generation, so that any researcher can verify, critique, or extend the analysis without relying on our word alone.

\end{document}
